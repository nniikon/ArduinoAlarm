\documentclass[12pt,a4paper]{article}
\usepackage[utf8]{inputenc}
\usepackage[russian]{babel}
\usepackage[T1]{fontenc}
\usepackage{hyperref}
\usepackage{geometry}
\geometry{left=2cm,right=2cm,top=2cm,bottom=2cm}
\usepackage{enumitem}
\usepackage{amssymb}

\title{Пояснительная записка к проекту}
% \date{\today}
\date{}

\begin{document}

\maketitle

\section{Состав проектной команды и вклад участников}

\begin{itemize}
    \item \textbf{Каспаров Николай Б01-304} - написание кода, проектирование деталей для печати.
    \item \textbf{Наумов Владимир Б01-303} - написание кода, сборка + пайка, оформление постера + ведение телеграм канала.
\end{itemize}

\section{Причины выбора проекта}

Актуальность проекта обусловлена проблемой своевременного пробуждения, характерной для студентов. Для борьбы с этим мы решили разработать будильник, который точно разбудит студента.

\section{Цель и задачи проекта}

\textbf{Цель проекта:} \\
Разработать будильник и протестировать его.
\\ \\
\textbf{Задачи проекта:}
\\
Создать будильник, чтобы он был:
\begin{enumerate}
    \item Дешёвым в производстве ($\leqslant$ 3 тыс.руб)
    \item Безотказным (Количество пропущенных сигналов должно быть $\sim$0)
    \item Способным обеспечивать точность до минут
\end{enumerate}
При постановке критериев мы отталкивались от потребностей целевого потребителя(т.е. студента).

\section{Описание устройства}

Будильник с мелодией и мощной лампой для пробуждения. Регулировка времени сигнала с точностью до минут. Для отображения времени - дисплей.\\
Размеры корпуса: 13x15x9 см. Питание от сети 220В.

\section{Описание процесса решения задач}

\begin{enumerate}
    \item Продуман функциональный дизайн устройства
    \item Куплены все комплектующие
    \item Сборка прототипа на макетной плате
    \item Написание кода для Arduino Nano
    \item Тестирование прототипа
    \item Создание и ведение телеграм канала
    \item Моделирование корпуса в CAD-системе для 3D-печати
    \item Печать корпуса
    \item Конечная сборка устройства
    \item Подготовка к сдаче проекта, написание документации и оформление презентации
\end{enumerate}

\textbf{Ссылки на цифровые объекты:}
\begin{itemize}
    \item \href{https://github.com/nniikon/ArduinoAlarm}{GitHub репозиторий проекта}
    \item \href{https://t.me/budilnikmipt}{Телеграм канал}
\end{itemize}

\section{Стоимость производства}

\begin{tabular}{|c|c|}
    \hline
    Компонента & Цена, руб \\ \hline
    Arduino Nano & 200 \\ \hline
    LCD 16x2 дисплей & 150 \\ \hline 
    RTC-модуль с батарейкой & 150 \\ \hline
    Лампа (100 Ватт) & 800 \\ \hline
    Реле (250В, 10А) & 300 \\ \hline 
    MP3 DFPlayer mini & 100 \\ \hline 
    SD-карта & 300 \\ \hline
    Динамик (3W, 4Ом) & 200 \\ \hline 
    Энкодер & 80 \\ \hline
    Понижающий преобразователь (220AC $\rightarrow$ 5DC) & 200 \\ \hline
    Итоговая стоимость & 2480 \\ \hline
\end{tabular}

\section{Анализ существующих аналогов и отличительные признаки проекта}

\href{https://www.ozon.ru/product/nochnik-umnyy-budilnik-imitiruet-voshod-solntsa-elektronnyy-budilnik-s-reguliruemoy-sensornoy-1627021430/}{\textbf{Самый мощный будильник с подсветкой с Ozon}}: \\
Дорогой(3.3 тыс. руб) + недостаточно мощный.
\\
\\
\href{https://www.philips.ru/c-m-hs/light-therapy}{\textbf{Philips Wake-Up Light}}: \\
Дорогой(9+ тыс. руб) + недостаточно мощный.
\\ \\
\textbf{Таким образом, наш будильник дешевле и значительно мощнее аналогов.}

\section{Описание процесса проектирования и изготовления продукта}

После установки основных критериев мы продумали функциональный дизайн и составили список комплектующих, необходимых для проекта. 

В качестве <<сердца>> будильника мы выбрали плату Arduino Nano. Такое решение мы приняли по большей степени из-за простоты прототипирования и написания кода. Так же мы выбрали LCD дисплей 16x2 с I2C адаптером, он полностью смог покрыть наше потребности в выводе текста. Для отсчёта времени использовался RTC-модуль с батарейкой. При создании функционального дизайна будильника было принято решение, что сигнал будет не только звуковым, но и световым. По этой причине мы взяли светодиодную лампу на 100 Вт и запитали её через реле (250В 10А с управляющим напряжением 5В). Для подачи звукового сигнала использовался MP3 DFPlayer mini + sd карта + Динамик 3W 4Ом. Для управления дисплеем был использован энкодер. Есть возможность выставлять время с точностью до минут. Для питания использовался понижающий преобразователь (220AC $\rightarrow$ 5DC).

После этого мы купили все комплектующие на AliExpress.

Когда пришли все комплектующие, мы разместили их на макетной плате.
Затем написали код для Arduino Nano (для написания кода использовали Arduino IDE). Параллельно с написанием кода мы тестировали его функциональность. В процессе этого обнаружили, что MP3 DFPlayer mini был бракованный, и нам пришлось заказать новый. Из-за ожидания мы потратили немало времени.

Корпус был смоделирован в CAD-системе <<Компас>> и слайсере <<PrusaSlicer>>. После этого напечатали на 3D-принтере из PLA пластика.

После того как все детали корпуса были напечатаны мы собрали устройство полностью и провели тестирование.

\section{Описание процесса тестирования и анализ результатов}

\textbf{Тестирование на функциональную корректность}:

Тестирование устройства проходило в условиях приближенным к реальным. Мы попросили знакомого студента помочь и объяснили ему как пользоваться нашим будильником. Вечером учебного дня он установил будильник на 8:30 следующего дня. На следующий день он сообщил о том, что устройство работает корректно и крайне положительно отозвался о его эффективности.
\\ \\
\textbf{Тестрирование на точность}:

Многократно устанавливали будильник, на разные промежутки времени. Погрешность составляет $\leqslant$ 1 минуты.
\\ \\
\textbf{По результатам тестирования можно с увереностью сказать что поставленные задачи были достигнуты.}

\end{document}